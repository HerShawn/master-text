% multiple1902 <multiple1902@gmail.com>
% conclusion.tex
% Copyright 2011~2012, multiple1902 (Weisi Dai)
% https://code.google.com/p/xjtuthesis/
%
% It is strongly recommended that you read documentations located at
%   http://code.google.com/p/xjtuthesis/wiki/Landing?tm=6
% in advance of your compilation if you have not read them before.
%
% This work may be distributed and/or modified under the
% conditions of the LaTeX Project Public License, either version 1.3
% of this license or (at your option) any later version.
% The latest version of this license is in
%   http://www.latex-project.org/lppl.txt
% and version 1.3 or later is part of all distributions of LaTeX
% version 2005/12/01 or later.
%
% This work has the LPPL maintenance status `maintained'.
%
% The Current Maintainer of this work is Weisi Dai.
%
\chapter{结论与展望}
\echapter{Conclusions and Future Work}
    \section{结论}
    \esection{Conclusions}
    连铸坯在高速、自动化的连续生产过程中,由于受原材料、轨制设备、系统控制等诸多技术因素的影响,导致连铸坯表面出现划伤、压痕、裂纹、结疤、刮伤等不同类型的缺陷。这些缺陷将会严重影响连铸坯的质量,使得连铸坯的抗腐蚀性、抗疲劳性、耐磨性等性能大大降低。这就要求连铸坯生产企业要对连铸坯进行充分的缺陷检测,保障连铸坯的质量,以减少在生产过程中可能发生的各类严重事故。因此,连铸坯表面的缺陷检测就成为连铸坯生产过程中极其重要的环节。本文的主要工作有:

    1)对现有的连铸坯表面缺陷检测方法进行了整理和总结,现有的连铸坯表面缺陷检测方法主要分为传统无损方法、基于RGB图像的方法和基于深度图像的方法。传统无损方法其主要使用一些钢板的物理特性,通过钢板对电、热、磁等反应的变化来探测缺陷,这些方法往往存在一些局限性,并且方法的实时性不高。基于RGB图像的方法是目前连铸坯表面缺陷检测的研究热点,这些方法主要是使用CCD相机采集钢板灰度图像,然后使用一些图像处理的手段来检测缺陷位置,然而这些算法对于伪缺陷的误检率较高。基于深度图像的方法主要是使用钢板的深度图像来检测缺陷,利用深度信息能够有效地检测出缺陷并排除伪缺陷。

    2)提出了一种基于显著性的缺陷检测方法,很好地解决了目前基于机器视觉的低对比度带黑斑缺陷检测难点,通过该方法,检测出的缺陷更加完整,误检数量明显减少,由于结合了改进的显著性检测方法,去除了图像中黑斑伪缺陷对检测结果的干扰,同时,该方法检测速度快,完全符合缺陷检测中对时间性能的要求。当然,该方法仍有继续研究和改进的空间,如何更加完整的提取到缺陷区域、更加完全的去除黑斑,以及时间性能的进一步提高都将成为接下来的工作重点。

    3)提出了一种基于基准面拟合的钢板深度图像缺陷检测算法。基于基准面的缺陷检测方法尝试去寻找钢板的基准面,然后将与基准面深度差别较大的区域作为缺陷种子点提取出来,最后使用基于种子点的图割方法来提取出完整的缺陷区域。另外,为了验证本文提出的两个缺陷检测算法,根据真实的钢板深度图像,我们使用人工合成的手段生成了一个钢板深度图像数据集,实验的结果显示,我们的方法达到了很好的效果。

    4)提出了一种基于法向量的深度图像缺陷检测算法。基于法向量的方法利用了缺陷部分的法向量变化较为剧烈这一特征,使用法向量梯度图像来刻画法向量的变化程度,概念上类似于灰度图像的梯度图,接着使用双阈值分割来获得候选缺陷区域,最终使用深度验证来确定候选缺陷区域是否为真正的缺陷。
    \section{展望}
    \esection{Future Work}
    本文一共提出了三种连铸坯钢板表面缺陷检测算法,其中,基于显著性图的缺陷检测算法是基于RGB图像的方法,基于基准面拟合的方法和基于法向量的方法是基于深度图像的方法。总结一下目前方法的不足之处,未来的工作主要集中在一下几点:

    1)对于基于显著性的方法,如何减少伪缺陷的误检率是关键,我们设计了一个流程来去除图像中的黑斑,但是对于钢板图像来说,伪缺陷除了黑斑还有氧化铁皮、水膜等等,如何改进算法使之对所有类型的伪缺陷都不产生误检同时对真正缺陷的检出率保持较高水平是未来的主要工作;

    2)需要收集连铸坯表面深度图像制作深度图像数据库,由于目前收集下载不到真实的连铸坯表面深度图像数据集,我们采用人工合成的手段制作了一个深度图像数据库,虽然在制作过程中我们尽可能地去拟合真实的钢板表面情况,但是在真正的连铸坯表面深度数据集上验证我们的算法的有效性还是有必要的;

    3)对于基于基准面拟合的方法,拟合出的基准面的质量直接决定了算法的缺陷检测性能,所以有必要对基准面拟合算法进一步研究和优化;

    4)对于基于法向量的方法,根据余弦相似度我们提出了一种法向量梯度图像来度量像素点之间法向量的差异,然而目前算法的查全率并不优秀,更改并设计一种新的度量方法来评价像素点之间法向量的差异是未来的主要研究内容。
